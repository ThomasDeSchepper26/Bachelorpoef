%%=============================================================================
%% Voorwoord
%%=============================================================================

\chapter*{\IfLanguageName{dutch}{Woord vooraf}{Preface}}%
\label{ch:voorwoord}

%% TODO:
%% Het voorwoord is het enige deel van de bachelorproef waar je vanuit je
%% eigen standpunt (``ik-vorm'') mag schrijven. Je kan hier bv. motiveren
%% waarom jij het onderwerp wil bespreken.
%% Vergeet ook niet te bedanken wie je geholpen/gesteund/... heeft

Met trots presenteer ik mijn bachelorproef, hiermee sluit ik het 3-jarig hoofdstuk af van de opleiding 
toegepaste informatica die ik volgde aan de Hogeschool Gent. Deze uitdagende opleiding heeft mij klaargestoomd 
voor een toekomst als systeembeheerder, en een fundamentele basis gelegd voor de rest van mijn carrière.

Binnen de wereld van IT wordt kun je automatisering bijna niet meer wegdenken. Binnen de richting toegepaste informatica wordt 
daar dan ook heel hard op ingezet, dat kun je merken aan de projecten en taken die we doorheen het traject uitvoerden. 
In deze projecten of taken hebben we kennis gemaakt met het belang van automatisatie, zeker voor systeembeheerders. 
Daarom vind ik het uiterst interessant om in het kader van automatisatie mijn onderzoek te doen.

Hoewel ik deze bachelorproef alleen heb uitgetypt, zijn er een aantal mensen die mij hebben ondersteund tijdens het proces.
Eerst en vooral wil ik dhr. Synaeve Pascal bedanken voor de hulp in het zoeken naar een onderzoeksvraag en voor 
het vertrouwen en de middelen die voor mij beschikbaar zijn gesteld om het onderzoek uit te voeren.

Daarnaast wil ook graag mijn co-promotor dhr. De Geyndt Robbe bedanken dat ik bij hem terecht kon voor technische vragen, 
voor zijn tijd en moeite.

Tot slot wil ook graag mijn promotor dhr. Van Maele Andy bedanken voor de feedback op het inhoudelijke aspect van mijn bachelorproef.

