%%=============================================================================
%% Samenvatting
%%=============================================================================

% TODO: De "abstract" of samenvatting is een kernachtige (~ 1 blz. voor een
% thesis) synthese van het document.
%
% Een goede abstract biedt een kernachtig antwoord op volgende vragen:
%
% 1. Waarover gaat de bachelorproef?
% 2. Waarom heb je er over geschreven?
% 3. Hoe heb je het onderzoek uitgevoerd?
% 4. Wat waren de resultaten? Wat blijkt uit je onderzoek?
% 5. Wat betekenen je resultaten? Wat is de relevantie voor het werkveld?
%
% Daarom bestaat een abstract uit volgende componenten:
%
% - inleiding + kaderen thema
% - probleemstelling
% - (centrale) onderzoeksvraag
% - onderzoeksdoelstelling
% - methodologie
% - resultaten (beperk tot de belangrijkste, relevant voor de onderzoeksvraag)
% - conclusies, aanbevelingen, beperkingen
%
% LET OP! Een samenvatting is GEEN voorwoord!

%%---------- Nederlandse samenvatting -----------------------------------------
%
% TODO: Als je je bachelorproef in het Engels schrijft, moet je eerst een
% Nederlandse samenvatting invoegen. Haal daarvoor onderstaande code uit
% commentaar.
% Wie zijn bachelorproef in het Nederlands schrijft, kan dit negeren, de inhoud
% wordt niet in het document ingevoegd.

\IfLanguageName{english}{%
\selectlanguage{dutch}
\chapter*{Samenvatting}
\lipsum[1-4]
\selectlanguage{english}
}{}

%%---------- Samenvatting -----------------------------------------------------
% De samenvatting in de hoofdtaal van het document

\chapter*{\IfLanguageName{dutch}{Samenvatting}{Abstract}}

In dit bachelorproef-voorstel zal onderzocht worden hoe het proces om nieuwe instanties met PimLayer te lan-
ceren kan worden geautomatiseerd zodat er geen kostbare tijd aan verspild wordt. Aware zoekt naar mogelijk-
heden om nieuwe klanten op een kostefficiënte en snelle manier op te zetten. In dit onderzoek wordt er gezocht
naar een oplossing om dit te realiseren aan de hand van Ansible en enkele scripts. Ansible is een automatiseringstool waarmee
je op een declaratieve wijze specifieke taken kunt beschrijven, die herbruikbaar zijn voor meerdere omgevin-
gen. Amazon is de grootste cloud aanbieder in de wereld en bied heel wat tools aan aan schappelijke prijzen. In
dit onderzoek wordt gebruik gemaakt van hun cloudservice AWS Lightsail om virtuele servers op te zetten die
daarna met Ansible geconfigureerd worden. De te verwachten resultaten zijn dat er een werkend herbruikbaar
playbook gebruikt kan worden om eenvoudig nieuwe klanten/instanties op te zetten.
